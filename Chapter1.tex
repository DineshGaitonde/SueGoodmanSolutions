\section*{Introduction To Point Set Topology}
\subsection*{Exercise 1.1.3}
If the complement of a subset $A$ of $\mathbb{R}^2$ is open, 
then $A$ is closed.

Consider a limit point $a$ of $A$. Assume $a \notin A$. So it is in
$A^c$, the complement of $A$. But $A^c$ is open. So I can create an open
set around $a$ completely contained in $A^c$. Since all the points in the
set so created are in $A^c$, none of them are in $A$. Hence $a$ is not a
limit point of $A$ contradicting our assumption that $a \notin A$. Hence
we have shown that all limit points of $A$ are contained in $A$. Hence
$A$ is closed.

\subsection*{Exercise 1.1.4}
\begin{enumerate}[(a)]
\item $A = \{\, \frac{1}{n} \mid n \in \mathbb{N} \,\}$
as a subset of the real line.

Consider $0$. It is not in the set, but is a limit point of $A$. So $A$
is not closed. Obviously $A$ is not open as well.

\item $B = \{\, (a,b) \mid 1 < a^2 + b^2 \le 2 \,\}$
as a subset of the plane $\mathbb{R}^2$.

Obviously not closed, since $(1,0)$ is a limit point, but is not in $B$.
Obviously not open, since cannot create an open set around $(2,0)$
completely contained in $B$.

\item $C = \{\, (a,b) \mid a \neq 0, b \neq 0 \,\}$
as a subset of the plane $\mathbb{R}^2$.

Obviously open. Limit points are the coordinate axes.

\item $D = \{\, (x, \sin{\frac{1}{x}}) \mid x > 0 \,\}$
as a subset of the plane $\mathbb{R}^2$.

Any point from $(0, -1)$ to $(0, 1)$ is a limit point. None of these
points are in $D$. So not closed. Since treated as a subset of 
$\mathbb{R}^2$, not open either.

\item The empty set $\emptyset$

Both open and closed. No limit points.
\end{enumerate}

\subsection*{Exercise 1.1.5}
What are the limit points of the set of points in $\mathbb{R}^2$ where
both coordinates are rational numbers.

Every point in $\mathbb{R}^2$ is a limit point. Any open set around any 
point (rational or irrational) has other rational points.

\subsection*{Exercise 1.1.7}
If a point $x$ is a limit point of set $A$ then there is a sequence of 
points $x_i$ none of which are equal to $x$, such that 
\[ \lim_{i\to\infty} x_i = x \]

Consider a family of open sets around $x$ indexed by $i$ where the 
$i^{th}$ open set is a ball of radius $\frac{1}{n}$ around $x$. Each open set includes some point in $A$ besides $x$ (definition of limit point). Pick an arbitrary point besides $x$ from this open set. We now have a sequence which converges to $x$.

\subsection*{Exercise 1.1.8}
Union of arbitrary open sets is open. Obvious.

\subsection*{Exercise 1.1.9}
\begin{enumerate}[(a)]
\item Intersection of finite collection of open sets is open.

Just prove for intersection of two open sets. Suppose we have two open
sets $U_1$ and $U_2$. If their intersection is empty, we are done. An
empty set is open by definition. Now suppose it is not empty. Consider a
point $x$ in the intersection. There exists a ball of radius $r_1$ around $x$ which is contained in $U_1$ and a ball of radius $r_2$ around $x$ 
which is contained in $U_2$. Let $r$ be the smaller of $r_1$ and $r_2$. A
ball of radius $r$ around $x$ is contained entirely in $U_1$ and in $U_2$,
So it is contained in the intersection of $U_1$ and $U_2$. So every 
point in the intersection can be covered with a ball entirely inside the
intersection. Hence the intersection is open.

\item Construct an infinite family of open sets in real line whose
intersection is not open.

Consider the collection of open intervals $(-\frac{1}{n}, \frac{1}{n})$
indexed by $n$. The intersection of this infinite collection of sets 
only includes the point $(0, 0)$ which is obviously not open.

\end{enumerate}

\subsection*{Exercise 1.1.10}
Define topology in terms of closed sets. Trivial.

\subsection*{Exercise 1.1.12}
Subset $C$ is closed in $A \in \mathbb{R}^n$ (subset topology)
if and only if $C = A\cap D$ for some closed $D$ in $\mathbb{R}^n$.

Subset topology is defined in terms of open set. Suppose $D^c$ is open
in $\mathbb{R}^n$. Then $A\cap D^c$ is open in $A$. Which means 
$(A\cap D^c)^c$ is closed in $A$. ({\bf Finish this proof})

